Introduction
*************

Dans la première version de mon simulateur, le choix avait été fait d'avoir un seul process par hôte. Des tests
approfondis ont montré que ça ne paraîssait pas une bonne solution. Voici un exemple de cas de figure qui pose problème:

2 noeuds intègrent le DST via un même contact:
 * Node 14 --> Node 121 --> Node 42 (intégration rapide, pas de scission requise)
   puis
 * Node 249 --> Node 121 --> Node 42 (scissions et ajout d'étage requis)

 - Node 121 reçoit bien ACK_CNX_REQ/14 de 42 mais le stocke parce qu'il est en attente de ACK_CNX_REQ/249 de 42

 - Node 14 ne recevant pas la réponse de 121 reste en 'b'. A ce stade de l'intégration, il fait déjà partie du DST
   et il reçoit donc ADD_STAGE/249 mais il ne peut y répondre ==> DEADLOCK

Un seul process (hébergé sur Node 121) est chargé ici de l'intégration de deux nouveaux noeuds différents et il attend
donc deux réponses. Le problème vient alors du fait qu'il ne peut pas traiter les réponses du premier pendant qu'il est occupé avec le deuxième.

Pour ne pas tenir compte de l'ordre de réception de ces réponses, on pense alors à deux solutions possibles:
1) Utiliser la fonction MSG_comm_waitany() de Simgrid
2) Utiliser plusieurs process dédiés :

J'ai choisi d'expérimenter la deuxième solution du fait qu'elle permet de bien séparer les tâches et donc, de présenter une trace bien plus lisible.
Voici l'idée générale :
  - chaque demande d'insertion de nouveau noeud est traitée par un process distinct créé pour l'occasion
  - un process fils ne peut pas recevoir de réponses à des requêtes qu'il n'a pas émises.

De plus, plusieurs nouveaux noeuds utilisant le même contact ne pouvant pas être traités simultanément,
un mécanisme de file d'attente pour les traiter séquentiellement a été mis en place.

Présentation d'une solution
***************************

Voici donc le schéma général de cette solution à process multiples:

<Call_procs.png>

launch_fork_process() est une fonction chargée de choisir comment exécuter les requêtes qui lui sont transmises : elle
les confie soit à un nouveau process (cas de CNX_REQ, BR_SPLIT, BR_CS_REQ), soit au process courant.

On utilise trois process par noeud:
    * Main_Proc:
      Comme son nom l'indique, c'est le process principal :
        - il se charge des initialisations et de la création du process Tasks_Queue (qui tourne tout le temps de la simulation).
        - lorsqu'il reçoit une requête, soit il la place dans la file tasks_queue (cas ces CNX_REQ), soit il la transmet à
          launch_fork_process().
        - il héberge les files tasks_queue (les CNX_REQ en attente) et delayed_tasks (les tâches différées)
        - il héberge ses propres files async_answers et sync_answers (les réponses asynchrones et synchrones attendues)

    * Tasks_Queue:
      Ce process est chargé de faire tourner la fonction run_tasks_queue() qui traite les files tasks_queue et
      delayed_tasks.
      Il héberge ses propres files async_answers et sync_answers.

    * Handle_Task:
      C'est ce process qui est éventuellement créé par launch_fork_process() pour traiter les requêtes concernées.
      Il héberge aussi ses propres files async_answers et sync_answers.

Dans le détail.
***************

run_tasks_queue
---------------

<Algo_run_tasks_queue.png>

La file tasks_queue contient toutes les demandes de connection (c'est à dire toutes les tâches CNX_REQ) reçues par le
noeud qui héberge cette file. run_tasks_queue() est donc chargée de traiter les requêtes présentes dans cette file, par ordre de
priorité.

Après avoir traité les tâches différées (voir détails plus loin), on s'occupe de la file tasks_queue proprement dite, à
condition d'être actif (à l'état 'a').

Chaque requête de cette file sera traitée par un process distinct, mais il n'est pas possible d'insérer plus d'un
nouveau noeud à la fois et il faut donc que ces process s'exécutent séquentiellement. C'est la raison d'être de la
variable de noeud Run_state : pour s'occuper de la tâche suivante, la tâche courante doit être terminée. (valeur IDLE)

Lors de l'arrivée d'un nouveau noeud, on a la séquence d'appels suivante : nouveau noeud --> contact --> leader. En cas
d'échec, le contact doit refaire une tentive plus tard, mais son leader peut avoir changé entre temps. Le leader doit
donc dépiler la requête alors que le contact doit la conserver. Deux valeurs de retour en cas d'échec sont alors
requises : un leader retourne la valeur FAILED alors qu'un contact retournera la valeur UPDATE_NOK.

Comme on peut le voir, une valeur de retour UPDATE_NOK laisse la requête dans la file pour une nouvelle tentative un peu
plus tard. ("Sleep for a while") Dans le cas contraire, on dépile pour passer à la requête suivante. (à noter qu'on
s'assure de toujours respecter l'ordre de priorité grâce au tri effectué à ce moment-là)

MAX_CNX ne sert qu'à afficher un message en cas de nombreuses tentatives d'exécution d'une même requête.

L'exécution d'une requête de la file est confiée à launch_fork_process() (voir plus haut)

run_delayed_tasks
-----------------


launch_fork_process
-------------------

<Algo_launch_fork_process.png>

Si la tâche transmise à cette fonction n'est ni une demande de connexion
(CNX_REQ), ni une diffusion de SPLIT ou de CS_REQ, alors elle est exécutée
localement, c'est à dire par le process courant.
Sinon, elle est confiée à un nouveau process créé pour l'occasion.

Remarque sur ce nouveau process:
--------------------------------
Il doit possèder ses propres file d'attente de réponses attendues (async_answers
et sync_answers) de sorte qu'elles ne puissent plus être mélangées avec celle
requises pour l'insertion d'un autre nouveau noeud.

Il est labelisé ainsi : xxx-nom_process-yyy où :
  - xxx est l'id du noeud courant
  - nom_process est le nom attribué par cette fonction (tel que "Cnx_req" ou "Br_split", etc)
  - yyy est l'id du nouveau noeud qui a déclenché la création de ce process.

Dans le cas où la tâche est une diffusion de CS_REQ (demande d'entrée en section
critique), on commence par s'assurer que le noeud courant est disponible (c'est
à dire répondrait favorablement à un CS_REQ) pour ne pas créer de process fils
inutilement.
